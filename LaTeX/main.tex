%%%%%%%%%%%%%%%%%%%%%%%%%%%%%%%%%%%%%%%%%%%%%%%%%%%%%%%
%
% IIT THESIS Main File
%%%%%%%%%%%%%%%%%%%%%%%%%%%%%%%%%%%%%%%%%%%%%%%%%%%%%%%
% File: main.tex                                   	  
% IIT THESIS Main File                                
% by Hongwei Jin on 05/30/2003                      
% Revised by Hongwei Jin on 11/18/2003                
%%%%%%%%%%%%%%%%%%%%%%%%%%%%%%%%%%%%%%%%%%%%%%%%%%%%%%%
%                          
% This file is a main LaTeX file for IIT thesis. 
% In this file, it will illustrate the structure of 
% the whole thesis. For detailed content, it will 
% include in other .tex files.
%%%%%%%%%%%%%%%%%%%%%%%%%%%%%%%%%%%%%%%%%%%%%%%%%%%%%%%
% Note: 
% 1. it is required to use iitthesis.cls style file;
% 2. it is required to include font11.clo, and font12.clo files;
% 3. it create a separate file mybib.bib for bibliography.
%%%%%%%%%%%%%%%%%%%%%%%%%%%%%%%%%%%%%%%%%%%%%%%%%%%%%%%



%% Include style file
\documentclass{iitthesis}

% Document Options:
%
% Note if you want to save paper when printing drafts,
% replace the above line by
%
%   \documentclass[draft]{iitthesis}
%
% See Help file for more about options.

\usepackage[dvips]{graphicx}    % This package is used for Figures
\usepackage{rotating}           % This package is used for landscape mode.
\usepackage{epsfig}
\usepackage{subfigure}          % These two packages, epsfig and subfigure, are used for creating subplots.
% Packages are explained in the Help document.
\usepackage[colorlinks,linkcolor=black]{hyperref}
\usepackage{listings}
%% Document content begins here

\begin{document}

%%% Declarations for Title Page %%%
	\title{Multi-stages stochastic programming on microgrid\\
	  and its mathematical simulations}
	\author{Hongwei Jin}
	\degree{Master of Science}
	\dept{Applied Mathematics}
	\date{May 2014}
	\copyrightnoticetrue      % crate copyright page or not
	%\coadvisortrue           % add co-advisor. activate it by removing % symbol to add co-advisor
	\maketitle                % create title and copyright pages
	
	\prelimpages         % Settings of preliminary pages are done with \prelimpages command

%%%  Acknowledgement
	\begin{acknowledgement}     % acknowledgment environment, this is optional
	Acknoledgement goes here.
	% or \input{acknowledgement.tex} % you need a separate acknowledgement.tex file to include it.
	\end{acknowledgement}

%%% Table of Contents
	\tableofcontents
	\clearpage

%%% List of Tables
	\listoftables
	\clearpage

%%%% List of Figures
	\listoffigures
	\clearpage

%%%% List of Symbols(optional)
	\listofsymbols
	\SymbolDefinition{$\beta$}{probability of non-detecting bad data}
	\SymbolDefinition{$\delta$}{Transition Coefficient Constant for the Design of Linear-Phase FIR Filters}
	\SymbolDefinition{$\zeta$}{Reflection Coefficient Parameter}
	\SymbolDefinition{$ C^R $}{Capacity of local generators}
	
	\clearpage
%%% Abstract 
	\begin{abstract}           % abstract environment, this is optional
	\par Your Abstract goes here!
	% or \documentclass{article}
\nopagenumbers{}
\begin{document}
\MakeUppercase{Thesis Title here}

Hongwei Jin, M.S.

Illinois Institute of Technology, May 2014

Adviser: Hemanshu Kaul

Abstract goes here.
\end{document}  %you need a separate abstract.tex file to include it.
	\end{abstract}
	\clearpage

\textpages     % Settings of text-pages are done with \textpages command

% Chapters are created with \Chapter{title} command
% Chapter 1 begins here
\Chapter{INTRODUCTION}
	\Section{Test} 
	\label{sec:int}
	
	This is explained in Section \ref{sec:int}. Now you will see a listing example:
	% An example for enumerate
	\begin{enumerate}
	  \item Suppression of hepatic glucose production
	  \item Stimulation of hepatic glucose uptake
	  \item Stimulation of glucose uptake by peripheral tissues,
	  mainly muscle
	\end{enumerate}
	
	\begin{quotation}
	This is a quotation. \cite{HK}
	\end{quotation}
	
	\Subsection{Test}
	
	Test

	\clearpage
	
% \Chapter{INTRODUCTION}

\Chapter{chapter 2}

test

\clearpage
%\Chapter{chapter 3}

test

\clearpage
%\Chapter{Simulations}

test

\clearpage
%\Chapter{Improvement and conclusion}

test

\clearpage

%
%%% APPENDIX
%

% Do the settings of appendices with \appendix command
\appendix
	
	% Then create each appendix using
	% \Appendix{title_of_appendix} command
	
	\Appendix{Table of Transition Coefficients for the Design of Linear-Phase FIR Filters}
	
	Your Appendix will go here !
	{\scriptsize \begin{lstlisting}
ClearAll["Global`*"]
t = RandomInteger[{0, 23}];
c = RandomChoice[{"N", "A", "M"}];
markovChainMatrix = {{0.7, 0.2, 0.1}, {0.6, 0.3, 0.1}, {0.5, 0.4, 
    0.1}};
buyPrice = 
  If[7 <= t <= 11 || 17 <= t <= 21, 0.099, 
   If[11 <= t <= 17, 0.081, 0.051]];
sellPrice = 0.8 * buyPrice ;
buyPriceTPlus = 
  If[7 <= t + 1 <= 11 || 17 <= t + 1 <= 21, 0.099, 
   If[11 <= t + 1 <= 17, 0.081, 0.051]];
sellPriceTPlus = 0.8*buyPriceTPlus;
avaiableResource = RandomInteger[{200, 400}];
demand = RandomInteger[{200, 500}];
iniInBat = RandomInteger[{0, 100}];
capBat = 100;
capRes = 400;
\end{lstlisting}}

Then solve to get the minimum cost:
{\scriptsize \begin{lstlisting}
Minimize[{
  (xGB + xGC)*
    buyPrice + (iniInBat + xGB + xRB - xBC - xBG)*0.01 + (iniInBat + 
     xGB + xRB + xBC + xBG) + 0.02 - (xBG + xRG)*sellPrice +
   If[c == "N", 0.7, 
     If[c == "A", 0.6, 
      0.5]]*((yGB1 + yGC1)*
       buyPriceTPlus + ((10 + xRB + xRB - xBC - xBG) + yGB1 + yRB1 - 
         yBC1 - yBG1)*0.01 - (yBG1 + yRG1)*sellPriceTPlus) +
   If[c == "N", 0.2, 
     If[c == "A", 0.3, 
      0.4]]*((yGB2 + yGC2)*
       buyPriceTPlus + ((10 + xRB + xRB - xBC - xBG) + yGB2 + yRB2 - 
         yBC2 - yBG2)*0.01 - (yBG2 + yRG2)*sellPriceTPlus) +
   If[c == "N", 0.1, 
     If[c == "A", 0.1, 
      0.1]]*((yGB3 + yGC3)*
       buyPriceTPlus + ((10 + xRB + xRB - xBC - xBG) + yGB3 + yRB3 - 
         yBC3 - yBG3)*0.01 - (yBG3 + yRG3)*sellPriceTPlus),
  xBC + xGC + xRC == demand &&
   0 <= iniInBat + xRB + xGB - xBG - xBC <= capBat &&
   xRG + xRB + xRC == avaiableResource &&
   xGB >= 0 && xGC >= 0 && xRB >= 0 && xRC >= 0 && xRG >= 0 && 
   xBC >= 0 && xBG >= 0 &&
   yGB1 >= 0 && yGC1 >= 0 && yRB1 >= 0 && yRC1 >= 0 && yRG1 >= 0 && 
   yBC1 >= 0 && yBG1 >= 0 &&
   yBC1 + yGC1 + yRC1 == demand + 100 &&
   -(iniInBat + xRB + xRB - xBC - xBG) <= 
    yRB1 + yGB1 - yBG1 - yBC1 <= 
    capBat - (iniInBat + xRB + xRB - xBC - xBG) &&
   yRG1 + yRB1 + yRC1 == avaiableResource &&
   yGB2 >= 0 && yGC2 >= 0 && yRB2 >= 0 && yRC2 >= 0 && yRG2 >= 0 && 
   yBC2 >= 0 && yBG2 >= 0 &&
   yBC2 + yGC2 + yRC2 == demand + 100 &&
   -(iniInBat + xRB + xRB - xBC - xBG) <= 
    yRB2 + yGB2 - yBG2 - yBC2 <= 
    capBat - (iniInBat + xRB + xRB - xBC - xBG) &&
   yRG2 + yRB2 + yRC2 == avaiableResource/2 &&
   yGB3 >= 0 && yGC3 >= 0 && yRB3 >= 0 && yRC3 >= 0 && yRG3 >= 0 && 
   yBC3 >= 0 && yBG3 >= 0 &&
   yBC3 + yGC3 + yRC3 == demand + 100 &&
   -(iniInBat + xRB + xRB - xBC - xBG) <= 
    yRB3 + yGB3 - yBG3 - yBC3 <= 
    capBat - (iniInBat + xRB + xRB - xBC - xBG) &&
   yRG3 + yRB3 + yRC3 == avaiableResource/4},
 {xGB, xGC, xRB, xRC, xRG, xBC, xBG, yGB1, yGC1, yRB1, yRC1, yRG1, 
  yBC1, yBG1, yGB2, yGC2, yRB2, yRC2, yRG2, yBC2, yBG2, yGB3, yGC3, 
  yRB3, yRC3, yRG3, yBC3, yBG3}]
\end{lstlisting}}
One example:
A random example like this, c="N", t=13
Then the result will be like this:
{\scriptsize \begin{lstlisting}
{39.855, {xGB -> 0., xGC -> 0., xRB -> 0., xRC -> 221., xRG -> 123., 
  xBC -> 0., xBG -> 0., yGB1 -> 0., yGC1 -> 0., yRB1 -> 23., 
  yRC1 -> 321., yRG1 -> 0., yBC1 -> 0., yBG1 -> 68., yGB2 -> 104., 
  yGC2 -> 0., yRB2 -> 0., yRC2 -> 172., yRG2 -> 0., yBC2 -> 149., 
  yBG2 -> 0., yGB3 -> 190., yGC3 -> 0., yRB3 -> 0., yRC3 -> 86., 
  yRG3 -> 0., yBC3 -> 235., yBG3 -> 0.}}
\end{lstlisting}
}
	
%	\moretoc
	
%	\Appendix{Name of your Second Appendix}
%	
%	Your second appendix text....
%	
%	\Appendix{Name of your Third Appendix}
%
%	Your third appendix text....
	
	
\bibliographystyle{plain}
\bibliography{mybib}

\end{document}